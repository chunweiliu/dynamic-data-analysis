\section{Discussion}
\label{sec:discussion}
This work proposed an automatic framework for processing a 4D CT image of pediatric airway and providing informative visualization for further analysis.
The current approach of data registration using detected landmarks has two issues.
First, the detection has outliers.
The outliers could be eliminated by examination the distance of each detected result between the detection mean.
Yet more powerful features might need to be developed.
Second, finding correspondents between a subject without enough landmarks to its statical atlas is handled by heuristics.
To fully solving this problem, reducing the rely on landmarks is necessary.
Since the within subject deformations were not dramatic, as Guerrero et al.'s suggestion in~\cite{guerrero2006dynamic}, a deformable image based registration within a subject would be a proper next step.

While the results has showed dynamic ranges and a comparison between dynamic data to normal control atlas, a more straightforward metric which quantifies atypicality of subjects would be very appreciated.
A reasonable next approach would be designed a metrics to evaluate the dynamics of airway.
A long term goal of this project is to develop a theory of 4D CT atlas building of pediatric airways and aim for going beyond to other parts of human body.