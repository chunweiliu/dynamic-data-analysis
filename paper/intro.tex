\section{Introduction}
\label{sec:intro}
% Why this problem is important?

% What 3D CT can do?
The analysis of pediatric airway geometry using computed tomography (CT) image has provided rich cues for doctors to diagnose respiratory issue for patients.
For example, given a 3D segmentation of CT image, doctors can diagnose tracheal stenosis by locating the position has smaller cross-sectional area in the airway and compare it to an normal control atlas~\cite{hong2014statistical}.

% What 4D CT can do?
Recently, dynamic CT data (4D CT) provides a better characterization of pediatric airways throughout the breathing cycle, for example to assess tracheomalacia, which is a disease of temporally collapse of partial airway.
Comparing with spirometry, analysis on 3D CT image provides additional information about {\it where} the respiratory issue might cause.
Then 4D CT image provides more information about {\it when} the respiratory issue might cause during the breath cycle.

% Where are we now?
However, how to preprocess the increasing amount of dynamic data and how to analysis these data are still open questions.
While standard techniques for analysis 3D CT could be applied on analysis 4D CT in a frame-by-frame fashion, two major challenges can be addressed as follow.
First, manually annotations or preprocessing cost of each subject has increased a factor proportional to the number of CT scans in a breath cycle.
Second, no information sharing between time steps might cause temporal inconsistence for the analysis.
In this work, I am going to address the above issues by performing dynamic data analysis using computer vision and machine learning approaches.

% Where should we go?
In Section~\ref{sec:methods}, I will introduce the methods applying on pediatric airway analysis.
Experiments on real airway data would be compiled in on Section~\ref{sec:experiments}.
In Section~\ref{sec:discussion}, I will discuss the future works, including building a 4D atlas for pediatric airways and extending the approaches to other dynamic data modalities.